% Created 2024-02-05 lun 10:00
% Intended LaTeX compiler: pdflatex
\documentclass[a4paper]{article}
\usepackage[utf8]{inputenc}
\usepackage[T1]{fontenc}
\usepackage{graphicx}
\usepackage{longtable}
\usepackage{wrapfig}
\usepackage{rotating}
\usepackage[normalem]{ulem}
\usepackage{amsmath}
\usepackage{amssymb}
\usepackage{capt-of}
\usepackage{hyperref}
\usepackage[spanish]{babel}
\usepackage[usenames,dvipsnames]{color} % Required for custom colors
\renewcommand{\ttdefault}{pcr} % MONOESPACIO CON NEGRIT
\usepackage{lastpage}
\usepackage{listings}
\usepackage{listingsutf8}
\renewcommand{\lstlistingname}{Listado}
\lstset{frame=single,inputencoding=utf8,basicstyle=\scriptsize\ttfamily,showstringspaces=false,numbers=none}
\definecolor{MyDarkGreen}{rgb}{0.0,0.4,0.0} % This is the color used for comments
\lstset{ breaklines=true, postbreak=\mbox{\textcolor{red}{$\hookrightarrow$}\space}, keywordstyle=\bfseries, keywordstyle=[1]\color{Blue}\bfseries,  keywordstyle=[2]\color{Purple}\bfseries,  keywordstyle=[3]\color{Blue}\underbar,   identifierstyle=,   commentstyle=\usefont{T1}{pcr}{m}{sl}\color{MyDarkGreen}\small,   stringstyle=\color{Purple},   showstringspaces=false,   tabsize=2,   morecomment=[l][\color{Blue}]{...} }
\lstset{literate=  {á}{{\'a}}1 {é}{{\'e}}1 {í}{{\'i}}1 {ó}{{\'o}}1 {ú}{{\'u}}1   {Á}{{\'A}}1 {É}{{\'E}}1 {Í}{{\'I}}1 {Ó}{{\'O}}1 {Ú}{{\'U}}1   {à}{{\`a}}1 {è}{{\`e}}1 {ì}{{\`i}}1 {ò}{{\`o}}1 {ù}{{\`u}}1   {À}{{\`A}}1 {È}{{\'E}}1 {Ì}{{\`I}}1 {Ò}{{\`O}}1 {Ù}{{\`U}}1   {ä}{{\"a}}1 {ë}{{\"e}}1 {ï}{{\"i}}1 {ö}{{\"o}}1 {ü}{{\"u}}1   {Ä}{{\"A}}1 {Ë}{{\"E}}1 {Ï}{{\"I}}1 {Ö}{{\"O}}1 {Ü}{{\"U}}1   {â}{{\^a}}1 {ê}{{\^e}}1 {î}{{\^i}}1 {ô}{{\^o}}1 {û}{{\^u}}1   {Â}{{\^A}}1 {Ê}{{\^E}}1 {Î}{{\^I}}1 {Ô}{{\^O}}1 {Û}{{\^U}}1   {œ}{{\oe}}1 {Œ}{{\OE}}1 {æ}{{\ae}}1 {Æ}{{\AE}}1 {ß}{{\ss}}1   {ű}{{\H{u}}}1 {Ű}{{\H{U}}}1 {ő}{{\H{o}}}1 {Ő}{{\H{O}}}1   {ç}{{\c c}}1 {Ç}{{\c C}}1 {ø}{{\o}}1 {å}{{\r a}}1 {Å}{{\r A}}1   {€}{{\euro}}1 {£}{{\pounds}}1 {«}{{\guillemotleft}}1   {»}{{\guillemotright}}1 {ñ}{{\~n}}1 {Ñ}{{\~N}}1 {¿}{{?`}}1 }
\usepackage{caption}
\usepackage{attachfile}
\usepackage[margin=1.5cm,includeheadfoot,includehead,includefoot]{geometry}
\hypersetup{colorlinks,linkcolor=black}
\usepackage{fancyhdr}
\pagestyle{fancyplain}
\chead{}
\lhead{}
\rhead{}
\cfoot{}
\lfoot{\begin{footnotesize}alvaro.gonzalezsotillo@educa.madrid.org\end{footnotesize}}
\rfoot{\begin{footnotesize}\thepage / \pageref{LastPage}\end{footnotesize}}
\usepackage[skins]{tcolorbox}
\usepackage{multicol}
\usepackage{changepage} %ajdustwidth
\usepackage{fancybox}
\usepackage{attachfile2}
\lhead{Extraordinaria 2021 (es el \\lhead)}
\rhead{Administración y Gestión de Bases de Datos (es el \\rhead)}
\lhead{Práctica creación de base de datos}
\rhead{Gestión de bases de datos}
\usepackage{svg}
\usepackage{letltxmacro}
\LetLtxMacro{\originalincludegraphics}{\includegraphics}
\renewcommand{\includegraphics}[2][]{\IfFileExists{#2.pdf}{\originalincludegraphics[#1]{#2.pdf}}{\originalincludegraphics[#1]{#2}}}
\LetLtxMacro{\originalincludesvg}{\includesvg}
\renewcommand{\includesvg}[2][]{\IfFileExists{#2.pdf}{\originalincludegraphics[#1]{#2.pdf}}{\originalincludegraphics[#1]{#2.svg.pdf}}}
\usepackage{comment}
\excludecomment{NOTES}
\author{Álvaro González Sotillo}
\date{\today}
\title{Diseño E/R, paso a SQL e inserción de datos}
\hypersetup{
 pdfauthor={Álvaro González Sotillo},
 pdftitle={Diseño E/R, paso a SQL e inserción de datos},
 pdfkeywords={},
 pdfsubject={},
 pdfcreator={Emacs 28.2 (Org mode 9.6.12)}, 
 pdflang={Spanish}}
\begin{document}

\maketitle
\setcounter{tocdepth}{1}
\tableofcontents

\captionsetup{font=scriptsize}

\setlength{\parindent}{0em}
\setlength{\parskip}{1em}

\newtcolorbox{Aviso}[1][Aviso]{
  enhanced,
  colback=gray!5!white,
  colframe=gray!75!black,fonttitle=\bfseries,
  colbacktitle=gray!85!black,
  attach boxed title to top left={yshift=-2mm,xshift=2mm},
  title=#1
}

\newtcolorbox{cuadrito}[1][Ignorado]{
  %drop shadow southeast,
  enhanced jigsaw,
  colback=white,
}


\newcommand{\StudentData}{
  \begin{cuadrito}[1\textwidth]
    \vspace{0.3cm}
    \large{
      \textbf{Apellidos:} \hrulefill \\
      \textbf{Nombre:} \hrulefill \\
      \textbf{Fecha:} \hrulefill \hspace{1cm} \textbf{Usuario:} \hrulefill \\
    }
    \vspace{-0.2cm}
  \end{cuadrito}
}


\section{Objetivo de la práctica}
\label{sec:org0000000}
Se pretende que el alumno sea capaz de pasar desde las necesidades de datos de una empresa hasta las órdenes SQL necesarias para implementar dicha necesidad en Oracle

Se puede descargar \href{https://alvarogonzalezsotillo.github.io/apuntes-clase/gestion-bd-asir1/apuntes/3/BD-3-EjercicioCreacionDeTablas-whatsapp.pdf}{la última versión de la práctica en este enlace}

\section{Modelo de datos}
\label{sec:org0000003}
Una compañía decide lanzar un sistema de mensajería al estilo de Whatsapp.
\begin{itemize}
\item De cada usuario se conoce su número de teléfono y un \emph{nick}, que podría repetirse.
\item Cada usuario mantiene una lista de contactos, que son otros usuarios.
\item Los usuarios pueden enviar mensajes de texto a cualquiera de sus contactos. Un mensaje es un texto de hasta 1000 caracteres.
\item Los mensajes tienen dos confirmaciones:
\begin{itemize}
\item Confirmación de recibido: el mensaje ha llegado al terminal del destinatario
\item Confirmación de leido: el usuario ha abierto el mensaje
\end{itemize}
\end{itemize}



\begin{itemize}
\item Los usuarios pueden crear grupos de \emph{chat}:
\begin{itemize}
\item Un grupo tiene un nombre.
\item El usuario creador es el administrador, y agrega a los usuarios que desea.
\item No se pueden crear dos grupos con el mismo nombre.
\item Un mensaje enviado a un grupo de \emph{chat} es recibido por todos sus integrantes (menos el que lo envía)
\item Las confirmaciones de los mensajes de un \emph{chat} son por cada usuario del \emph{chat}
\end{itemize}
\end{itemize}


\section{Realizar el diagrama \textbf{ER} con Oracle \textbf{SQLDeveloper} (2 puntos)}
\label{sec:org0000006}
En el diagrama deben ser visibles (al menos):
\begin{itemize}
\item Los nombres de las entidades
\item Los atributos de las entidades
\item Las relaciones entre las entidades
\end{itemize}




\section{Crear las tablas de Oracle utilizando \textbf{SQL} (1 punto)}
\label{sec:org0000009}
Las tablas se crearán en el servidor \textbf{Oracle} del instituto.

Se recomienda mantener un fichero de texto de extensión \texttt{.SQL} con las órdenes \textbf{SQL} de creación de tablas, sus claves primarias, extranjeras, valores por defecto y restricciones, por si dicho servidor fallase.


\section{Insertar datos en las tablas (1 punto)}
\label{sec:org000000c}

El cliente quiere insertar los siguientes datos para comprobar la idoneidad del modelo relacional:

\begin{Aviso}
Estos datos no están todos normalizados. Si alguna tabla de la base de datos coincide con esta estructura, podría ser incorrecta.
Este apartado solo puede valorarse si las tablas están en \textbf{tercera forma normal}.
\end{Aviso}

\begin{center}
\begin{tabular}{|r|l|}
\hline
\textbf{Usuarios} & \\[0pt]
\hline
Teléfono & Nick\\[0pt]
\hline
11111111 & Pepe\\[0pt]
22222222 & María\\[0pt]
33333333 & Juan\\[0pt]
44444444 & Susana\\[0pt]
\hline
\end{tabular}
\begin{tabular}{|l|l|}
\hline
\textbf{Contactos} & \\[0pt]
\hline
Nick & Sus contactos\\[0pt]
\hline
María & Pepe\\[0pt]
Pepe & María, Juan, Susana\\[0pt]
Juan & Pepe, Susana\\[0pt]
Susana & Juan\\[0pt]
\hline
\end{tabular}
\end{center}

\begin{center}
\begin{tabular}{|l|l|l|l|l|}
\hline
\textbf{Mensajes} &  &  &  & \\[0pt]
\hline
Remitente & Destinatario & Recibido & Leído & Mensaje\\[0pt]
\hline
Pepe & María & Sí & Sí & Vamos a hacer un grupo de ASIR\\[0pt]
María & Pepe & Sí & Sí & Vale\\[0pt]
Susana & Juan & Sí & No & ¿Te apuntas a la nieve?\\[0pt]
Juan & Susana & No & No & Me apunto\\[0pt]
\hline
\end{tabular}
\end{center}

\begin{center}
\begin{tabular}{|l|l|l|}
\hline
\textbf{\emph{Chats}} &  & \\[0pt]
\hline
Nombre de \emph{chat} & Administrador & Integrantes adicionales\\[0pt]
\hline
Cosas de ASIR & Pepe & María, Juan, Susana\\[0pt]
Viaje a la nieve & Juan & Pepe, Susana\\[0pt]
\hline
\end{tabular}
\end{center}

\begin{center}
\begin{tabular}{|l|l|l|l|l|}
\hline
\textbf{Mensajes en \emph{chats}} &  &  &  & \\[0pt]
\hline
Nombre de \emph{chat} & Remitente & Mensaje & Recibido & Leido\\[0pt]
\hline
Cosas de Asir & Pepe & Hoy el profesor Iñaki no viene & Recibido por todos & Leído por todos\\[0pt]
Cosas de Asir & María & Pues ya estoy en el insti :( & Recibido por todos & Nadie lo ha leído\\[0pt]
Viaje a la nieve & Juan & El sábado no puedo & Recibido solo por Pepe & Leido solo por Pepe\\[0pt]
\hline
\end{tabular}
\end{center}

Es importante que los datos coincidan con los especificados. Los valores \emph{cierto} y \emph{falso} se pondrán como un número con \texttt{1} y \texttt{0} (esto puede facilitar la vista \texttt{V\_MENSAJESPORGRUPO}).

\section{Consultas (8 puntos)}
\label{sec:org000000f}

Para comprobar los nuevos datos introducidos, se crearán varias vistas. Una vista basada sobre una tabla que no esté en 3FN no será valorada.

\begin{itemize}
\item \texttt{V\_CONTACTOS} (1.5 punto)
\begin{itemize}
\item \texttt{(nombreusuario, nombrecontacto)}
\end{itemize}
\item \texttt{V\_MENSAJESENVIADOSA} (2.5 puntos)
\begin{itemize}
\item \texttt{(nombredestinatario, nombreemisor, textomensaje, recibido, leido)}
\item Todos los mensajes enviados a un destinatario, sean directos o dentro de un grupo de \emph{chat}
\end{itemize}
\item \texttt{V\_MENSAJESPORGRUPO} (2.5 puntos)
\begin{itemize}
\item \texttt{(nombrechat, nombreemisor, textomensaje, cuantosrecibido, cuantosleido)}
\item Mensajes enviados a un grupo de \emph{chat}
\end{itemize}
\item \texttt{V\_GRUPOSDECHAT} (1.5 punto)
\begin{itemize}
\item \texttt{(nombrechat,cuantosintegrantes)}
\item Número de integrantes de un chat, incluido el creador
\end{itemize}
\end{itemize}

\begin{lstlisting}[language=SQL,label= ,caption={Código de creación de vistas},captionpos=b,firstnumber=1,numbers=left]
  create view V_CONTACTOS(nombreusuario, nombrecontacto) as
  select ...
  
  create view V_MENSAJESENVIADOSA(nombredestinatario,nombreemisor,textomensaje,recibido,leido) as
  select ...
  
  create view V_MENSAJESPORGRUPO(nombrechat,nombreemisor,textomensaje,cuantosrecibido,cuantosleido) as
  select ...
    
  create view V_GRUPOSDECHAT(nombrechat,cuantosintegrantes) as
  select ...
\end{lstlisting}


\section{Instrucciones de entrega}
\label{sec:org0000012}
\begin{itemize}
\item El ejercicio se realizará y entregará de manera individual.
\begin{itemize}
\item Solo se admiten trabajos en pareja, si en clase es necesario compartir ordenador.
\end{itemize}
\item Entrega el modelo ER en formato PDF en la tarea del aula virtual
\item En tu \href{https://aulavirtual3.educa.madrid.org/ies.alonsodeavellan.alcala/mod/assign/view.php?id=29534}{usuario del servidor de Oracle del instituto}, crea las tablas, inserta los datos y define las vistas.
\item La corrección se realizará de forma semiautomática. Es \textbf{importante} que los nombres de vistas y atributos de vistas sean los especificados.
\item Presta atención al plazo de entrega (con fecha y hora).
\end{itemize}


\begin{Aviso}
El servidor está compartido con todos los compañeros, con otros grupos y otros profesores. Es posible que las bases de datos allí guardadas se pierdan.

Se recomienda guardar todas las órdenes (\texttt{create table}, \texttt{insert}, \texttt{create view}, \ldots) en ficheros de texto.
\end{Aviso}
\end{document}
