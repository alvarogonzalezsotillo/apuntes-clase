%%%%%%%%%%%%%%%%%%%%%%%%%%%%%%%%%%%%%%%%%
% Programming/Coding Assignment
% LaTeX Template
%
% This template has been downloaded from:
% http://www.latextemplates.com
%
% Original author:
% Ted Pavlic (http://www.tedpavlic.com)
%
% Note:
% The \lipsum[#] commands throughout this template generate dummy text
% to fill the template out. These commands should all be removed when 
% writing assignment content.
%
% This template uses a Perl script as an example snippet of code, most other
% languages are also usable. Configure them in the "CODE INCLUSION 
% CONFIGURATION" section.
%
%%%%%%%%%%%%%%%%%%%%%%%%%%%%%%%%%%%%%%%%%

%----------------------------------------------------------------------------------------
%	PACKAGES AND OTHER DOCUMENT CONFIGURATIONS
%----------------------------------------------------------------------------------------

\documentclass[a4paper]{article}
\usepackage[utf8]{inputenc}
\usepackage{listingsutf8}
\usepackage[spanish]{babel}




\usepackage{fancyhdr} % Required for custom headers
\usepackage{lastpage} % Required to determine the last page for the footer
\usepackage{extramarks} % Required for headers and footers
\usepackage[usenames,dvipsnames]{color} % Required for custom colors
\usepackage{graphicx} % Required to insert images
\usepackage{listings} % Required for insertion of code
\usepackage{courier} % Required for the courier font
\usepackage{lipsum} % Used for inserting dummy 'Lorem ipsum' text into the template
\usepackage{changepage}   % for the adjustwidth environment
\usepackage{svg}
\usepackage{attachfile2}
\usepackage{currfile}
\usepackage{multicol}
\usepackage{alltt}
\usepackage{framed}
\usepackage{caption}
\usepackage{eurosym}
\usepackage{seqsplit}
\usepackage{tabularx}
\usepackage{xcomment}

\hypersetup{
colorlinks,
citecolor=black,
filecolor=black,
linkcolor=black,
urlcolor=blue
}


% Margins
\topmargin=-0.45in
\evensidemargin=0in
\oddsidemargin=0in
\textwidth=6.5in
\textheight=9.8in
\headsep=0.25in
\linespread{1.1} % Line spacing
\setlength{\parindent}{0em}
\setlength{\parskip}{1em}


% Set up the header and footer
\pagestyle{fancy}
%\lhead{\hmwkAuthorName} % Top left header
\lhead{\hmwkClass}
\rhead{\hmwkTitle} % Top right header
\chead{} % Top center head
\lfoot{\lastxmark} % Bottom left footer
\cfoot{} % Bottom center footer
\rfoot{ \thepage\ / \protect\pageref{LastPage}} % Bottom right footer
\renewcommand\headrulewidth{0.4pt} % Size of the header rule
\renewcommand\footrulewidth{0.4pt} % Size of the footer rule



%----------------------------------------------------------------------------------------
%	CODE INCLUSION CONFIGURATION
%----------------------------------------------------------------------------------------
\renewcommand{\ttdefault}{pcr}

%----------------------------------------------------------------------------------------
%	DOCUMENT STRUCTURE COMMANDS
%	Skip this unless you know what you're doing
%----------------------------------------------------------------------------------------

% Header and footer for when a page split occurs within a problem environment
\newcommand{\enterProblemHeader}[1]{
%\nobreak\extramarks{#1}{#1 continued on next page\ldots}\nobreak
%\nobreak\extramarks{#1 (continued)}{#1 continued on next page\ldots}\nobreak
}

% Header and footer for when a page split occurs between problem environments
\newcommand{\exitProblemHeader}[1]{
%\nobreak\extramarks{#1 (continued)}{#1 continued on next page\ldots}\nobreak
%\nobreak\extramarks{#1}{}\nobreak
}

\setcounter{secnumdepth}{0} % Removes default section numbers
\newcounter{homeworkProblemCounter} % Creates a counter to keep track of the number of problems

\newcommand{\homeworkProblemName}{}
\newenvironment{homeworkProblem}[1][]{ % Makes a new environment called homeworkProblem which takes 1 argument (custom name) but the default is "Problem #"
\stepcounter{homeworkProblemCounter} % Increase counter for number of problems
\renewcommand{\homeworkProblemName}{Ejercicio \arabic{homeworkProblemCounter} #1} % Assign \homeworkProblemName the name of the problem
\section{\homeworkProblemName} % Make a section in the document with the custom problem count
\enterProblemHeader{\homeworkProblemName} % Header and footer within the environment
}{
\exitProblemHeader{\homeworkProblemName} % Header and footer after the environment
%\clearpage
}


\newcommand{\problemAnswer}[1]{ % Defines the problem answer command with the content as the only argument
\noindent\framebox[\columnwidth][c]{\begin{minipage}{0.98\columnwidth}#1\end{minipage}} % Makes the box around the problem answer and puts the content inside
}

\newcommand{\homeworkSectionName}{}
\newenvironment{homeworkSection}[1]{ % New environment for sections within homework problems, takes 1 argument - the name of the section
\renewcommand{\homeworkSectionName}{#1} % Assign \homeworkSectionName to the name of the section from the environment argument
\subsection{\homeworkSectionName} % Make a subsection with the custom name of the subsection
\enterProblemHeader{\homeworkProblemName\ [\homeworkSectionName]} % Header and footer within the environment
}{
\enterProblemHeader{\homeworkProblemName} % Header and footer after the environment
}

%----------------------------------------------------------------------------------------
%	NAME AND CLASS SECTION
%----------------------------------------------------------------------------------------

\newcommand{\hmwkTitle}{Redefine el hmwkTitle} % Assignment title
\newcommand{\hmwkDueDate}{asdfadsf} % Due date
\newcommand{\hmwkClass}{Redefine hmwkClass} % Course/class
\newcommand{\hmwkClassTime}{} % Class/lecture time
\newcommand{\hmwkClassInstructor}{} % Teacher/lecturer
\newcommand{\hmwkAuthorName}{Álvaro González Sotillo} % Your name

%----------------------------------------------------------------------------------------
%	TITLE PAGE
%----------------------------------------------------------------------------------------

\title{
\vspace{2in}
\textmd{\textbf{\hmwkClass:\ \hmwkTitle}}\\
\vspace{0.1in}\large{\textit{\hmwkClassInstructor\ \hmwkClassTime}}
\vspace{3in}
}

\author{\textbf{\hmwkAuthorName}}
\date{} % Insert date here if you want it to appear below your name


%----------------------------------------------------------------------------------------

\usepackage{fancybox}
\newcommand{\codigo}[1]{\texttt{#1}}


% CUADRITO
\newsavebox{\fmboxx}
\newenvironment{cuadritoantiguo}[1][14cm]
{\noindent \begin{center} \begin{lrbox}{\fmboxx}\noindent\begin{minipage}{#1}}
{\end{minipage}\end{lrbox}\noindent\shadowbox{\usebox{\fmboxx}} \end{center}}






\newenvironment{entradasalida}[2][14cm]
{
  \newcommand{\elnombredelafiguradeentradasalida}{#2}
  \begin{figure}[h]
    \begin{cuadrito}[#1]
      \begin{scriptsize}
\begin{alltt}
}
{%
\end{alltt}%
      \end{scriptsize}%
    \end{cuadrito}%
    \caption{\elnombredelafiguradeentradasalida}
  \end{figure}
}


\newenvironment{entradasalidacols}[2][14cm]
{
\newcommand{\elnombredelafiguradeentradasalida}{#2}
%\begin{wrapfigure}{}{0.1}
\begin{cuadrito}[#1]
\begin{scriptsize}
\begin{alltt}
}
{
\end{alltt}
\end{scriptsize}
\end{cuadrito}
\captionof{figure}{\elnombredelafiguradeentradasalida}
%\end{wrapfigure}
}


\usepackage{pgffor}
\newcommand{\ficheroautoref}[0]{%
  \foreach \ficherotex in {../../../common/plantilla-ejercicio.tex,../../common/plantilla-ejercicio.tex,../common/plantilla-ejercicio.tex,../apuntes/common/plantilla-ejercicio.tex}{

    \IfFileExists{\ficherotex}%
    {%
      \textattachfile[mimetype=text/plain,%
      author={plantilla-ejercicio.tex},
      description={La plantilla},%
      subject={La plantilla}]%
      {\ficherotex}%
      {}%
      % RECUPERO ESPACIO VERTICAL PERDIDO
      \vspace{-6em}%
    }%
    {}%
  }
  

  \IfFileExists{\currfilename}
  {
    \textattachfile[mimetype=text/plain,
    author={\currfilename},
    description={El fichero TEX original para crear este documento, no sea que se nos pierda},
    subject={El fichero TEX original para crear este documento, no sea que se nos pierda}]
    {\currfilename}
    {}
  }
  {}
}

\newcommand{\entradausuario}[1]{\textit{\textbf{#1}}}

\newcommand{\enlace}[2]{\textcolor{blue}{{\href{#1}{#2}}}}

\newcommand{\adjuntarfichero}[3]{
\textattachfile[mimetype=text/plain,
color={0 0 0},
description={#3},
subject={#1}]
{#1}
{\textcolor{blue}{\codigo{#2}}}
}

\newcommand{\adjuntardoc}[2]{%
\textattachfile[mimetype=text/plain,%
color={0 0 0},%
description={#1},%
subject={#1}]%
{#1}%
{\textcolor{blue}{#2}}%
}%


\newcommand{\plantilladeclase}[2]{
\adjuntarfichero{#1.java}{#2}{Plantilla para la clase #2}
}

% https://en.wikibooks.org/wiki/LaTeX/Source_Code_Listings#Encoding_issue
\lstset{literate=
  {á}{{\'a}}1 {é}{{\'e}}1 {í}{{\'i}}1 {ó}{{\'o}}1 {ú}{{\'u}}1
  {Á}{{\'A}}1 {É}{{\'E}}1 {Í}{{\'I}}1 {Ó}{{\'O}}1 {Ú}{{\'U}}1
  {à}{{\`a}}1 {è}{{\`e}}1 {ì}{{\`i}}1 {ò}{{\`o}}1 {ù}{{\`u}}1
  {À}{{\`A}}1 {È}{{\'E}}1 {Ì}{{\`I}}1 {Ò}{{\`O}}1 {Ù}{{\`U}}1
  {ä}{{\"a}}1 {ë}{{\"e}}1 {ï}{{\"i}}1 {ö}{{\"o}}1 {ü}{{\"u}}1
  {Ä}{{\"A}}1 {Ë}{{\"E}}1 {Ï}{{\"I}}1 {Ö}{{\"O}}1 {Ü}{{\"U}}1
  {â}{{\^a}}1 {ê}{{\^e}}1 {î}{{\^i}}1 {ô}{{\^o}}1 {û}{{\^u}}1
  {Â}{{\^A}}1 {Ê}{{\^E}}1 {Î}{{\^I}}1 {Ô}{{\^O}}1 {Û}{{\^U}}1
  {œ}{{\oe}}1 {Œ}{{\OE}}1 {æ}{{\ae}}1 {Æ}{{\AE}}1 {ß}{{\ss}}1
  {ű}{{\H{u}}}1 {Ű}{{\H{U}}}1 {ő}{{\H{o}}}1 {Ő}{{\H{O}}}1
  {ç}{{\c c}}1 {Ç}{{\c C}}1 {ø}{{\o}}1 {å}{{\r a}}1 {Å}{{\r A}}1
  {€}{{\euro}}1 {£}{{\pounds}}1 {«}{{\guillemotleft}}1
  {»}{{\guillemotright}}1 {ñ}{{\~n}}1 {Ñ}{{\~N}}1 {¿}{{?`}}1
}

\lstset{
  inputencoding=utf8,
  captionpos=b,
  frame=single,
  basicstyle=\small\ttfamily,
  showstringspaces=false,
  numbers=none,
  numbers=left,
  xleftmargin=2em,
  breaklines=true,
  postbreak=\mbox{\textcolor{red}{$\hookrightarrow$}\space}
}

\renewcommand{\lstlistingname}{Listado}
\captionsetup[lstlisting]{font={footnotesize},labelfont=bf,position=bottom}
\captionsetup[figure]{font={footnotesize},labelfont=bf}
\lstnewenvironment{listadojava}[2]
{
  \lstset{language=Java,caption={#1},label={#2}}
  \noindent\minipage{\linewidth}%
}
{\endminipage}

% LISTADO SHELL
\lstnewenvironment{listadoshell}[2]
{
  \lstset{language=bash,caption={#1},label={#2}}
  \noindent\minipage{\linewidth}%
}
{\endminipage}

% LISTADO TXT
\lstnewenvironment{listadotxt}[2]
{
  \lstset{caption={#1},label={#2}, keywords={}}
  \noindent\minipage{\linewidth}%
}
{\endminipage}

% LISTADO SQL
\lstnewenvironment{listadosql}[2]%
{%
  %(el 1 )caption es #1\\%
  %(el 2) label es #2\\%
  \lstset{language=sql,caption={#1},label={#2},morekeywords={if,end}}%
  \noindent\minipage{\linewidth}%
}
{\endminipage}
  






\newcommand{\graficosvg}[3][14cm]{
\begin{figure}[htbp]
\centering
\textattachfile{#2.svg}{
\color{black}
\includesvg[width=#1]{#2}
}
\caption{#3}
\end{figure}
}


\newcommand{\graficosvguml}[3][7cm]{
  \texttt{\graficosvg[#1]{#2}{#3}}
}


\newcommand{\primerapagina}{
\newpage
\ficheroautoref
\tableofcontents
\newpage
}

% CAJAS
\usepackage[skins]{tcolorbox}
\newtcolorbox{Aviso}[1][Aviso]{
  enhanced,
  colback=gray!5!white,
  colframe=gray!75!black,fonttitle=\bfseries,
  colbacktitle=gray!85!black,
  attach boxed title to top left={yshift=-2mm,xshift=2mm},
  title=#1
}

\newtcolorbox{cuadrito}[1][Ignorado]{
  %drop shadow southeast,
  enhanced jigsaw,
  colback=white,
}

\newcommand{\StudentData}{
  \begin{cuadrito}[1\textwidth]
    \vspace{0.3cm}
    \large{
      \textbf{Apellidos:} \hrulefill \\
      \textbf{Nombre:} \hrulefill \\
      \textbf{Fecha:} \hrulefill \hspace{1cm} \textbf{Grupo:} \hrulefill \\
    }
    \vspace{-0.2cm}
  \end{cuadrito}
}

% TEXTO EN MONOESPACIO PERO QUE SE PARTE EN LINEAS
\usepackage{seqsplit}
\newcommand{\tecnico}[1]{\texttt{\seqsplit{#1}}}

% REEMPLAZAR TEXTO, NO LO USO
\def\replace#1#2#3{%
 \def\tmp##1#2{##1#3\tmp}%
   \tmp#1\stopreplace#2\stopreplace}
\def\stopreplace#1\stopreplace{}

% PARA QUE SALGA EN imenu
\newcommand{\imenu}[1]{#1}
\newcommand{\imenutwo}[1]{#1}
