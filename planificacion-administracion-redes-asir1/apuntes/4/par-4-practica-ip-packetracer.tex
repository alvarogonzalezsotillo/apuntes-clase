% Created 2024-01-18 jue 12:54
% Intended LaTeX compiler: pdflatex
\documentclass[a4paper]{article}
\usepackage[utf8]{inputenc}
\usepackage[T1]{fontenc}
\usepackage{graphicx}
\usepackage{longtable}
\usepackage{wrapfig}
\usepackage{rotating}
\usepackage[normalem]{ulem}
\usepackage{amsmath}
\usepackage{amssymb}
\usepackage{capt-of}
\usepackage{hyperref}
\usepackage[spanish]{babel}
\usepackage[usenames,dvipsnames]{color} % Required for custom colors
\renewcommand{\ttdefault}{pcr} % MONOESPACIO CON NEGRIT
\usepackage{lastpage}
\usepackage{listings}
\usepackage{listingsutf8}
\renewcommand{\lstlistingname}{Listado}
\lstset{frame=single,inputencoding=utf8,basicstyle=\scriptsize\ttfamily,showstringspaces=false,numbers=none}
\definecolor{MyDarkGreen}{rgb}{0.0,0.4,0.0} % This is the color used for comments
\lstset{ breaklines=true, postbreak=\mbox{\textcolor{red}{$\hookrightarrow$}\space}, keywordstyle=\bfseries, keywordstyle=[1]\color{Blue}\bfseries,  keywordstyle=[2]\color{Purple}\bfseries,  keywordstyle=[3]\color{Blue}\underbar,   identifierstyle=,   commentstyle=\usefont{T1}{pcr}{m}{sl}\color{MyDarkGreen}\small,   stringstyle=\color{Purple},   showstringspaces=false,   tabsize=2,   morecomment=[l][\color{Blue}]{...} }
\lstset{literate=  {á}{{\'a}}1 {é}{{\'e}}1 {í}{{\'i}}1 {ó}{{\'o}}1 {ú}{{\'u}}1   {Á}{{\'A}}1 {É}{{\'E}}1 {Í}{{\'I}}1 {Ó}{{\'O}}1 {Ú}{{\'U}}1   {à}{{\`a}}1 {è}{{\`e}}1 {ì}{{\`i}}1 {ò}{{\`o}}1 {ù}{{\`u}}1   {À}{{\`A}}1 {È}{{\'E}}1 {Ì}{{\`I}}1 {Ò}{{\`O}}1 {Ù}{{\`U}}1   {ä}{{\"a}}1 {ë}{{\"e}}1 {ï}{{\"i}}1 {ö}{{\"o}}1 {ü}{{\"u}}1   {Ä}{{\"A}}1 {Ë}{{\"E}}1 {Ï}{{\"I}}1 {Ö}{{\"O}}1 {Ü}{{\"U}}1   {â}{{\^a}}1 {ê}{{\^e}}1 {î}{{\^i}}1 {ô}{{\^o}}1 {û}{{\^u}}1   {Â}{{\^A}}1 {Ê}{{\^E}}1 {Î}{{\^I}}1 {Ô}{{\^O}}1 {Û}{{\^U}}1   {œ}{{\oe}}1 {Œ}{{\OE}}1 {æ}{{\ae}}1 {Æ}{{\AE}}1 {ß}{{\ss}}1   {ű}{{\H{u}}}1 {Ű}{{\H{U}}}1 {ő}{{\H{o}}}1 {Ő}{{\H{O}}}1   {ç}{{\c c}}1 {Ç}{{\c C}}1 {ø}{{\o}}1 {å}{{\r a}}1 {Å}{{\r A}}1   {€}{{\euro}}1 {£}{{\pounds}}1 {«}{{\guillemotleft}}1   {»}{{\guillemotright}}1 {ñ}{{\~n}}1 {Ñ}{{\~N}}1 {¿}{{?`}}1 }
\usepackage{caption}
\usepackage{attachfile}
\usepackage[margin=1.5cm,includeheadfoot,includehead,includefoot]{geometry}
\hypersetup{colorlinks,linkcolor=black}
\usepackage{fancyhdr}
\pagestyle{fancyplain}
\chead{}
\lhead{}
\rhead{}
\cfoot{}
\lfoot{\begin{footnotesize}alvaro.gonzalezsotillo@educa.madrid.org\end{footnotesize}}
\rfoot{\begin{footnotesize}\thepage / \pageref{LastPage}\end{footnotesize}}
\usepackage[spanish]{babel}
\usepackage[usenames,dvipsnames]{color} % Required for custom colors
\renewcommand{\ttdefault}{pcr} % MONOESPACIO CON NEGRIT
\usepackage{lastpage}
\usepackage{listings}
\usepackage{listingsutf8}
\renewcommand{\lstlistingname}{Listado}
\lstset{frame=single,inputencoding=utf8,basicstyle=\scriptsize\ttfamily,showstringspaces=false,numbers=none}
\definecolor{MyDarkGreen}{rgb}{0.0,0.4,0.0} % This is the color used for comments
\lstset{ breaklines=true, postbreak=\mbox{\textcolor{red}{$\hookrightarrow$}\space}, keywordstyle=\bfseries, keywordstyle=[1]\color{Blue}\bfseries,  keywordstyle=[2]\color{Purple}\bfseries,  keywordstyle=[3]\color{Blue}\underbar,   identifierstyle=,   commentstyle=\usefont{T1}{pcr}{m}{sl}\color{MyDarkGreen}\small,   stringstyle=\color{Purple},   showstringspaces=false,   tabsize=2,   morecomment=[l][\color{Blue}]{...} }
\lstset{literate=  {á}{{\'a}}1 {é}{{\'e}}1 {í}{{\'i}}1 {ó}{{\'o}}1 {ú}{{\'u}}1   {Á}{{\'A}}1 {É}{{\'E}}1 {Í}{{\'I}}1 {Ó}{{\'O}}1 {Ú}{{\'U}}1   {à}{{\`a}}1 {è}{{\`e}}1 {ì}{{\`i}}1 {ò}{{\`o}}1 {ù}{{\`u}}1   {À}{{\`A}}1 {È}{{\'E}}1 {Ì}{{\`I}}1 {Ò}{{\`O}}1 {Ù}{{\`U}}1   {ä}{{\"a}}1 {ë}{{\"e}}1 {ï}{{\"i}}1 {ö}{{\"o}}1 {ü}{{\"u}}1   {Ä}{{\"A}}1 {Ë}{{\"E}}1 {Ï}{{\"I}}1 {Ö}{{\"O}}1 {Ü}{{\"U}}1   {â}{{\^a}}1 {ê}{{\^e}}1 {î}{{\^i}}1 {ô}{{\^o}}1 {û}{{\^u}}1   {Â}{{\^A}}1 {Ê}{{\^E}}1 {Î}{{\^I}}1 {Ô}{{\^O}}1 {Û}{{\^U}}1   {œ}{{\oe}}1 {Œ}{{\OE}}1 {æ}{{\ae}}1 {Æ}{{\AE}}1 {ß}{{\ss}}1   {ű}{{\H{u}}}1 {Ű}{{\H{U}}}1 {ő}{{\H{o}}}1 {Ő}{{\H{O}}}1   {ç}{{\c c}}1 {Ç}{{\c C}}1 {ø}{{\o}}1 {å}{{\r a}}1 {Å}{{\r A}}1   {€}{{\euro}}1 {£}{{\pounds}}1 {«}{{\guillemotleft}}1   {»}{{\guillemotright}}1 {ñ}{{\~n}}1 {Ñ}{{\~N}}1 {¿}{{?`}}1 }
\usepackage{caption}
\usepackage{attachfile}
\usepackage[margin=1.5cm,includeheadfoot,includehead,includefoot]{geometry}
\hypersetup{colorlinks,linkcolor=black}
\usepackage{fancyhdr}
\pagestyle{fancyplain}
\chead{}
\lhead{}
\rhead{}
\cfoot{}
\lfoot{\begin{footnotesize}alvaro.gonzalezsotillo@educa.madrid.org\end{footnotesize}}
\rfoot{\begin{footnotesize}\thepage / \pageref{LastPage}\end{footnotesize}}
\usepackage[skins]{tcolorbox}
\usepackage{multicol}
\usepackage{changepage} %ajdustwidth
\usepackage{fancybox}
\usepackage{attachfile2}
\lhead{Extraordinaria 2021 (es el \\lhead)}
\rhead{Administración y Gestión de Bases de Datos (es el \\rhead)}
\lhead{Red IP simple \textit{Packet Tracer}}
\rhead{Planificación y administración de redes}
\usepackage{svg}
\usepackage{letltxmacro}
\LetLtxMacro{\originalincludegraphics}{\includegraphics}
\renewcommand{\includegraphics}[2][]{\IfFileExists{#2.pdf}{\originalincludegraphics[#1]{#2.pdf}}{\originalincludegraphics[#1]{#2}}}
\LetLtxMacro{\originalincludesvg}{\includesvg}
\renewcommand{\includesvg}[2][]{\IfFileExists{#2.pdf}{\originalincludegraphics[#1]{#2.pdf}}{\originalincludegraphics[#1]{#2.svg.pdf}}}
\usepackage{comment}
\excludecomment{NOTES}
\usepackage{svg}
\usepackage{letltxmacro}
\LetLtxMacro{\originalincludegraphics}{\includegraphics}
\renewcommand{\includegraphics}[2][]{\IfFileExists{#2.pdf}{\originalincludegraphics[#1]{#2.pdf}}{\originalincludegraphics[#1]{#2}}}
\LetLtxMacro{\originalincludesvg}{\includesvg}
\renewcommand{\includesvg}[2][]{\IfFileExists{#2.pdf}{\originalincludegraphics[#1]{#2.pdf}}{\originalincludegraphics[#1]{#2.svg.pdf}}}
\usepackage{comment}
\excludecomment{NOTES}
\author{Álvaro González Sotillo Álvaro González Sotillo}
\date{\today}
\title{Red IP simple \emph{Packet Tracer}\\\medskip
\large  }
\hypersetup{
 pdfauthor={Álvaro González Sotillo Álvaro González Sotillo},
 pdftitle={Red IP simple \emph{Packet Tracer}},
 pdfkeywords={ },
 pdfsubject={ },
 pdfcreator={Emacs 28.2 (Org mode 9.6.12)}, 
 pdflang={Spanish}}
\begin{document}

\maketitle
\setcounter{tocdepth}{1}
\tableofcontents

\captionsetup{font=scriptsize}

\captionsetup{font=scriptsize}

\setlength{\parindent}{0em}
\setlength{\parskip}{1em}

\newtcolorbox{Aviso}[1][Aviso]{
  enhanced,
  colback=gray!5!white,
  colframe=gray!75!black,fonttitle=\bfseries,
  colbacktitle=gray!85!black,
  attach boxed title to top left={yshift=-2mm,xshift=2mm},
  title=#1
}

\newtcolorbox{cuadrito}[1][Ignorado]{
  %drop shadow southeast,
  enhanced jigsaw,
  colback=white,
}


\newcommand{\StudentData}{
  \begin{cuadrito}[1\textwidth]
    \vspace{0.3cm}
    \large{
      \textbf{Apellidos:} \hrulefill \\
      \textbf{Nombre:} \hrulefill \\
      \textbf{Fecha:} \hrulefill \hspace{1cm} \textbf{Usuario:} \hrulefill \\
    }
    \vspace{-0.2cm}
  \end{cuadrito}
}


\section{Objetivo de la práctica}
\label{sec:org0000000}
Tras la práctica, se espera que el alumno haya conseguido:
\begin{itemize}
\item Familiarizarse con la interfaz de \emph{Packet Tracer}
\item Dividir adecuadamente una red en varias subredes
\end{itemize}

Está dispobile la \href{https://alvarogonzalezsotillo.github.io/apuntes-clase/planificacion-administracion-redes-asir1/apuntes/4/par-4-practica-ip-packetracer.pdf}{última versión de la práctica en este enlace}.



\section{Descripción del problema}
\label{sec:org0000009}
Una empresa tiene disponible la red \texttt{10.n.80.0/22}. La utilizará para dar servicio a un nuevo edificio con 4 plantas. Cada planta pertenecerá a una red IP distinta.
\begin{itemize}
\item 1ª planta: Se prevén 100 puestos de trabajo, cada uno con su ordenador
\item 2ª planta: 150 puestos de trabajo
\item 3ª planta: 30 puestos de trabajo
\item 4ª planta: 200 puestos de trabajo
\item Red de routers: Cada planta tendrá un router. Los routers se unirán entre sí en otra subred de la red original, con capacidad para 25 routers.
\end{itemize}


El número \texttt{n} de la red se corresponde con el asignado al alumno en  \href{http://bit.ly/2sji9gQ}{http://bit.ly/2sji9gQ}

\subsection{Características del \emph{subnetting}}
\label{sec:org0000003}
\begin{itemize}
\item Las redes tendrán el tamaño mínimo posible dados los hosts que contienen
\item Las parte no asignada de la red original se compodrá de redes contiguas tan grandes como sea posible
\end{itemize}

\subsection{Características del diseño \emph{Packet Tracer}}
\label{sec:org0000006}
\begin{itemize}
\item Cada planta tendrá un \emph{router}, su \emph{switch} (sistema horizontal), el primer host y el último host.
\item Los routers de las plantas se utirán por un switch (sistema vertical).
\item Habrá comunicación a nivel IP (\texttt{ping})
\begin{itemize}
\item entre todos los hosts de una planta
\item entre todos los routers, usando la red de routers
\end{itemize}
\item No se considera un error que no se pueda hacer \texttt{ping} entre plantas
\end{itemize}



\section{Qué se valorará}
\label{sec:org000000c}
Se entregará un fichero PKT con el contenido de la práctica, y un fichero de texto (PDF, DOCX, ODT) con el resultado del subnetting, incluyendo las redes que quedan libres. Todo ello, dentro de un fichero ZIP.

Se tendrá en cuenta:
\begin{itemize}
\item La corrección técnica
\item La claridad
\item La apariencia profesional: gramática, ortografía, homogeneidad \ldots
\end{itemize}

\section{Instrucciones de entrega}
\label{sec:org000000f}
\begin{itemize}
\item El ejercicio se realizará y entregará de manera individual.
\begin{itemize}
\item Solo se admiten trabajos en pareja, si en clase es necesario compartir ordenador.
\end{itemize}
\item Sube el documento a \href{https://aulavirtual3.educa.madrid.org/ies.alonsodeavellan.alcala/course/view.php?id=189}{la tarea correspondiente en el aula virtual}
\item Presta atención al plazo de entrega (con fecha y hora).
\end{itemize}
\end{document}
