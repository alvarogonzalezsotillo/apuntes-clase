% Created 2024-09-08 dom 13:12
% Intended LaTeX compiler: pdflatex
\documentclass[a4paper]{article}
\usepackage[utf8]{inputenc}
\usepackage[T1]{fontenc}
\usepackage{graphicx}
\usepackage{longtable}
\usepackage{wrapfig}
\usepackage{rotating}
\usepackage[normalem]{ulem}
\usepackage{amsmath}
\usepackage{amssymb}
\usepackage{capt-of}
\usepackage{hyperref}
\usepackage[spanish]{babel}
\usepackage[usenames,dvipsnames]{color} % Required for custom colors
\renewcommand{\ttdefault}{pcr} % MONOESPACIO CON NEGRIT
\usepackage{lastpage}
\usepackage{listings}
\usepackage{listingsutf8}
\renewcommand{\lstlistingname}{Listado}
\lstset{frame=single,inputencoding=utf8,basicstyle=\scriptsize\ttfamily,showstringspaces=false,numbers=none}
\definecolor{MyDarkGreen}{rgb}{0.0,0.4,0.0} % This is the color used for comments
\lstset{ breaklines=true, postbreak=\mbox{\textcolor{red}{$\hookrightarrow$}\space}, keywordstyle=\bfseries, keywordstyle=[1]\color{Blue}\bfseries,  keywordstyle=[2]\color{Purple}\bfseries,  keywordstyle=[3]\color{Blue}\underbar,   identifierstyle=,   commentstyle=\usefont{T1}{pcr}{m}{sl}\color{MyDarkGreen}\small,   stringstyle=\color{Purple},   showstringspaces=false,   tabsize=2,   morecomment=[l][\color{Blue}]{...} }
\lstset{literate=  {á}{{\'a}}1 {é}{{\'e}}1 {í}{{\'i}}1 {ó}{{\'o}}1 {ú}{{\'u}}1   {Á}{{\'A}}1 {É}{{\'E}}1 {Í}{{\'I}}1 {Ó}{{\'O}}1 {Ú}{{\'U}}1   {à}{{\`a}}1 {è}{{\`e}}1 {ì}{{\`i}}1 {ò}{{\`o}}1 {ù}{{\`u}}1   {À}{{\`A}}1 {È}{{\'E}}1 {Ì}{{\`I}}1 {Ò}{{\`O}}1 {Ù}{{\`U}}1   {ä}{{\"a}}1 {ë}{{\"e}}1 {ï}{{\"i}}1 {ö}{{\"o}}1 {ü}{{\"u}}1   {Ä}{{\"A}}1 {Ë}{{\"E}}1 {Ï}{{\"I}}1 {Ö}{{\"O}}1 {Ü}{{\"U}}1   {â}{{\^a}}1 {ê}{{\^e}}1 {î}{{\^i}}1 {ô}{{\^o}}1 {û}{{\^u}}1   {Â}{{\^A}}1 {Ê}{{\^E}}1 {Î}{{\^I}}1 {Ô}{{\^O}}1 {Û}{{\^U}}1   {œ}{{\oe}}1 {Œ}{{\OE}}1 {æ}{{\ae}}1 {Æ}{{\AE}}1 {ß}{{\ss}}1   {ű}{{\H{u}}}1 {Ű}{{\H{U}}}1 {ő}{{\H{o}}}1 {Ő}{{\H{O}}}1   {ç}{{\c c}}1 {Ç}{{\c C}}1 {ø}{{\o}}1 {å}{{\r a}}1 {Å}{{\r A}}1   {€}{{\euro}}1 {£}{{\pounds}}1 {«}{{\guillemotleft}}1   {»}{{\guillemotright}}1 {ñ}{{\~n}}1 {Ñ}{{\~N}}1 {¿}{{?`}}1 }
\usepackage{caption}
\usepackage{attachfile2}
\usepackage[margin=1.5cm,includeheadfoot,includehead,includefoot]{geometry}
\hypersetup{colorlinks,linkcolor=black}
\usepackage{fancyhdr}
\pagestyle{fancyplain}
\chead{}
\lhead{}
\rhead{}
\cfoot{}
\lfoot{\begin{footnotesize}alvaro.gonzalezsotillo@educa.madrid.org\end{footnotesize}}
\rfoot{\begin{footnotesize}\thepage / \pageref{LastPage}\end{footnotesize}}
\usepackage{svg}
\usepackage{letltxmacro}
\LetLtxMacro{\originalincludegraphics}{\includegraphics}
\renewcommand{\includegraphics}[2][]{\IfFileExists{#2.pdf}{\originalincludegraphics[#1]{#2.pdf}}{\originalincludegraphics[#1]{#2}}}
\LetLtxMacro{\originalincludesvg}{\includesvg}
\renewcommand{\includesvg}[2][]{\IfFileExists{#2.pdf}{\originalincludegraphics[#1]{#2.pdf}}{\originalincludegraphics[#1]{#2.svg.pdf}}}
\usepackage{comment}
\excludecomment{NOTES}
\author{Álvaro González Sotillo}
\date{\today}
\title{IMPLANTACIÓN DE APLICACIONES WEB\\\medskip
\large  (CÓDIGO: 0376)}
\hypersetup{
 pdfauthor={Álvaro González Sotillo},
 pdftitle={IMPLANTACIÓN DE APLICACIONES WEB},
 pdfkeywords={ 0376},
 pdfsubject={},
 pdfcreator={Emacs 29.1.90 (Org mode 9.6.10)}, 
 pdflang={Spanish}}
\begin{document}

\maketitle
\setcounter{tocdepth}{1}
\tableofcontents

\captionsetup{font=scriptsize}

\section{Cómo serán las clases}
\label{sec:org0000000}
\begin{itemize}
\item Teoría
\begin{itemize}
\item Basada en apuntes
\item Con un libro de texto
\end{itemize}
\item Ejercicios
\begin{itemize}
\item Se realizan en clase o en casa
\item Se ponen en común al día siguiente
\end{itemize}
\item Práctica
\begin{itemize}
\item Máquinas virtuales
\item Servidor del profesor
\item Posiblemente, una nube
\end{itemize}
\item Trabajos
\end{itemize}

\section{Materiales}
\label{sec:org0000006}
\begin{itemize}
\item Memoria USB
\item Correo electrónico
\item Acceso a Internet fuera del aula
\item Portátil propio (opcional)
\begin{itemize}
\item No se puede usar la red cableada del centro
\item Se usará la Wifi
\end{itemize}
\end{itemize}
\subsection{Libro de texto}
\label{sec:org0000003}
\begin{itemize}
\item Aun no se ha decidido cuál
\end{itemize}

\section{Entrega de Trabajos}
\label{sec:org0000009}
\begin{itemize}
\item Via \textbf{Moodle}
\begin{itemize}
\item Nuestro curso es \url{https://aulavirtual3.educa.madrid.org/ies.alonsodeavellan.alcala}
\item El curso es accesible incluso sin usuario, inicialmente
\end{itemize}
\item Se utilizará Microsoft Office (\textbf{DOC}, \textbf{DOCX})
\begin{itemize}
\item Opcionalmente, \textbf{PDF} o LibreOffice (\textbf{ODT})
\end{itemize}
\item Se tendrá en cuenta
\begin{itemize}
\item La corrección técnica de los trabajos
\item La fecha de entrega
\item Expresión, sintaxis, ortografía
\item La apariencia profesional
\end{itemize}
\end{itemize}


\section{Normas}
\label{sec:org000000f}
\begin{itemize}
\item Retrasos y faltas
\item Uso de los ordenadores
\begin{itemize}
\item No pueden utilizarse para tareas distintas de las encargadas por el profesor
\item Se respetará a otros alumnos
\end{itemize}
\item Móviles
\begin{itemize}
\item No.
\item Un \emph{smartwatch} se considera un móvil.
\end{itemize}
\end{itemize}


\subsection{Averías de los ordenadores}
\label{sec:org000000c}
\begin{itemize}
\item Los problemas se comunican al profesor en cuanto se detectan
\item Se deben hacer copias de seguridad para no perder los datos de los discos
\begin{itemize}
\item Pen Drive
\item Disco Externo
\item Correos enviados a uno mismo
\item Copias en los ordenadores de otros compañeros
\end{itemize}
\item Norma fundamental:
\end{itemize}
\textbf{Si se pierde porque no hay copia, es que no era importante}

\section{Cómo será la evaluación}
\label{sec:org0000012}
\begin{itemize}
\item Las notas de las evaluaciones (1ª,2ª,3ª) no son realmente importantes
\item Solo interesa la nota de la evaluación final
\item Basado en \emph{Resultados de aprendizaje} (RA)
\begin{itemize}
\item Cada RA supone un porcentaje de la nota final
\item Cada prueba (examen, trabajo) indicará que RA evalúa, en qué porcentaje
\item Se necesita aprobar cada RA para aprobar el módulo
\end{itemize}
\end{itemize}

\section{Actividades}
\label{sec:org0000015}
\begin{itemize}
\item Trabajos
\item Actitud
\begin{itemize}
\item Puntualidad, interés, preguntas al profesor, puesta en común de resultados, comportamiento\ldots{}
\end{itemize}
\item Exámenes
\item Examen final evaluación ordinaria
\begin{itemize}
\item Con los RA no superados
\end{itemize}
\item Examen evaluación extraordinaria
\begin{itemize}
\item Incluirá todos los RA. La nota del examen será la nota del módulo.
\end{itemize}
\item Entrega de trabajos
\begin{itemize}
\item Individuales, o por parejas si se comparte ordenador
\item Un trabajo entregado fuera de plazo tiene una nota máxima de \emph{6}
\end{itemize}
\end{itemize}



\section{Contenidos}
\label{sec:org0000018}

Según el Decreto 12/2010, de 18 de marzo


\begin{itemize}
\item Conceptos generales de la arquitectura aplicaciones web:
\begin{itemize}
\item Aplicaciones web vs. aplicaciones de escritorio.
\item Arquitectura cliente servidor. Elementos.
\item Arquitectura de tres niveles.
\item Protocolos de aplicación más usados: HTTP (Hyper Text Transfer Protocol), HTTPS (Hyper Text Transfer Protocol Secure), FTP.
\end{itemize}
\end{itemize}
\begin{itemize}
\item Instalación de servidores de aplicaciones web:
\begin{itemize}
\item Análisis de requerimientos:
\begin{itemize}
\item Del equipo servidor: procesador, memoria, almacenamiento, tolerancia a fallos\ldots{}
\item Del sistema operativo anfitrión: sistema de ficheros\ldots{}
\item Del propio servidor de aplicaciones: tiempos de respuesta, conexiones concurrentes\ldots{}
\item Del sistema gestor de bases de datos: accesos concurrentes
\item De las conexiones de red; internet, intranet, medios físicos\ldots{}
\end{itemize}
\item Sistema operativo anfitrión: instalación y configuración.
\item Servidor web: instalación y configuración.
\item Sistema gestor de bases de datos: instalación y configuración.
\item Procesamiento de código: lenguajes de script en cliente y servidor.
\item Módulos y componentes necesarios.
\item Utilidades de prueba e instalación integrada (paquetes que incluyan el servidor web, el lenguaje de script del servidor y el sistema gestor).
\item Verificación del funcionamiento integrado.
\item Documentación de la instalación.
\end{itemize}
\end{itemize}
\begin{itemize}
\item Instalación de gestores de contenidos:
\begin{itemize}
\item Conceptos generales y casuística de uso recomendado.
\item Tipos de gestores de contenidos: portales, de enseñanza, blogs, wikis, foros\ldots{}
\item Licencias de uso.
\item Requerimientos de funcionamiento: servidor web, lenguaje de script, sistema gestor de bases de datos,\ldots{}
\item Instalación.
\item Creación de la base de datos.
\item Estructura.
\item Creación de contenidos.
\item Personalización de la interfaz.
\item Mecanismos de seguridad integrados: control de acceso, usuarios\ldots{}
\item Verificación del rendimiento y funcionamiento.
\item Publicación.
\end{itemize}
\end{itemize}
\begin{itemize}
\item Administración de gestores de contenidos:
\begin{itemize}
\item Usuarios y grupos.
\item Perfiles.
\item Seguridad. Control de accesos.
\item Integración de módulos.
\item Gestión de temas.
\item Plantillas.
\item Copias de seguridad.
\item Sindicación de contenidos.
\item Importación y exportación de la información.
\end{itemize}
\end{itemize}
\begin{itemize}
\item Adaptación de gestores de contenidos:
\begin{itemize}
\item Selección de modificaciones a realizar.
\item Reconocimiento de elementos involucrados.
\item Modificación de la apariencia.
\item Incorporación y adaptación de funcionalidades.
\item Verificación del funcionamiento.
\item Documentación.
\end{itemize}
\end{itemize}
\begin{itemize}
\item Implantación de aplicaciones de ofimática web:
\begin{itemize}
\item Tipos de aplicaciones.
\item Requerimientos.
\item Instalación.
\item Configuración.
\item Integración de aplicaciones heterogéneas.
\item Gestión de usuarios.
\item Control de accesos.
\item Aseguramiento de la información.
\end{itemize}
\end{itemize}
\begin{itemize}
\item Diseño del contenido y la apariencia de documentos web:
\begin{itemize}
\item Lenguajes de marcas para representar el contenido de un documento:
\item Modificación de la apariencia de un documento web con hojas de estilos.
\end{itemize}
\end{itemize}
\begin{itemize}
\item Programación de documentos web utilizando lenguajes de «script» del cliente:
\begin{itemize}
\item Diferencias entre la ejecución en lado del cliente y del servidor.
\item Modelo de objetos del documento DOM.
\item Resolución de problemas concretos:
\begin{itemize}
\item Validación de formularios.
\item Introducción de comportamientos dinámicos. Captura de eventos.
\end{itemize}
\item Limitaciones y riesgos de ataques.
\end{itemize}
\end{itemize}
\begin{itemize}
\item Programación de documentos web utilizando lenguajes de «script» de servidor:
\begin{itemize}
\item Clasificación.
\item Integración con los lenguajes de marcas.
\item Sintaxis.
\item Herramientas de edición de código.
\item Elementos del lenguaje estructurado: tipos de datos, variables, operadores, estructuras de control, subprogramas\ldots{}
\item Elementos de orientación a objeto.
\item Comentarios.
\item Funciones integradas y de usuario.
\item Gestión de errores.
\item Mecanismos de introducción de información: formularios. Procesamiento de datos recibidos desde el cliente.
\item Métodos de envío de datos desde el cliente al servidor.
\item Autenticación de usuarios.
\item Control de accesos.
\item Sesiones. Mecanismos para mantener el estado entre conexiones.
\item Configuración del intérprete.
\end{itemize}
\end{itemize}
\begin{itemize}
\item Acceso a bases de datos desde lenguajes de «script» de servidor:
\begin{itemize}
\item Integración de los lenguajes de «script» de servidor con los sistemas gestores de bases de datos.
\item Conexión a bases de datos. Acceso mediante funciones nativas. Acceso mediante ODBC (Open DataBase Connectivity).
\item Creación de bases de datos y tablas.
\item Creación de vistas. Creación de procedimientos almacenados.
\item Recuperación de la información de la base de datos desde una página web.
\item Modificación de la información almacenada: inserciones, actualizaciones y borrados.
\item Verificación de la información.
\item Gestión de errores.
\item Verificación del funcionamiento y pruebas de rendimiento.
\item Mecanismos de seguridad y control de accesos.
\item Documentación.
\end{itemize}
\end{itemize}









\section{Criterios de evaluación}
\label{sec:org0000033}

\subsection{1. Prepara el entorno de desarrollo y los servidores de aplicaciones Web instalando e integrando las funcionalidades necesarias.}
\label{sec:org000001b}
Criterios de evaluación:
\begin{enumerate}
\item Se ha identificado el software necesario para su funcionamiento.
\item Se han identificado las diferentes tecnologías empleadas.
\item Se han instalado y configurado servidores Web y de bases de datos.
\item Se han reconocido las posibilidades de procesamiento en los entornos cliente y servidor.
\item Se han añadido y configurado los componentes y módulos necesarios para el procesamiento de código en el servidor.
\item Se ha instalado y configurado el acceso a bases de datos.
\item Se ha establecido y verificado la seguridad en los accesos al servidor.
\item Se han utilizado plataformas integradas orientadas a la prueba y desarrollo de aplicaciones Web.
\item Se han documentado los procedimientos realizados.
\end{enumerate}
\subsection{2. Implanta gestores de contenidos seleccionándolos y estableciendo la configuración de sus parámetros.}
\label{sec:org000001e}
Criterios de evaluación:
\begin{enumerate}
\item Se ha valorado el uso y utilidad de los gestores de contenidos.
\item Se han clasificado según la funcionalidad principal del sitio Web que permiten gestionar.
\item Se han instalado diferentes tipos de gestores de contenidos.
\item Se han diferenciado sus características (uso, licencia, entre otras).
\item Se han personalizado y configurado los gestores de contenidos.
\item Se han activado y configurado los mecanismos de seguridad proporcionados por los propios gestores de contenidos.
\item Se han realizado pruebas de funcionamiento.
\item Se han publicado los gestores de contenidos.
\end{enumerate}
\subsection{3. Administra gestores de contenidos adaptándolos a los requerimientos y garantizando la integridad de la información.}
\label{sec:org0000021}
Criterios de evaluación:
\begin{enumerate}
\item Se han adaptado y configurado los módulos del gestor de contenidos.
\item Se han creado y gestionado usuarios con distintos perfiles.
\item Se han integrado módulos atendiendo a requerimientos de funcionalidad.
\item Se han realizado copias de seguridad de los contenidos.
\item Se han importado y exportado contenidos en distintos formatos.
\item Se han gestionado plantillas.
\item Se han integrado funcionalidades de sindicación.
\item Se han realizado actualizaciones.
\item Se han obtenido informes de acceso.
\end{enumerate}
\subsection{4. Gestiona aplicaciones de ofimática Web integrando funcionalidades y asegurando el acceso a la información.}
\label{sec:org0000024}
Criterios de evaluación:
\begin{enumerate}
\item Se ha reconocido la utilidad de las aplicaciones de ofimática Web.
\item Se han clasificado según su funcionalidad y prestaciones específicas.
\item Se han instalado aplicaciones de ofimática Web.
\item Se han configurado las aplicaciones para integrarlas en una intranet.
\item Se han gestionado las cuentas de usuario.
\item Se han aplicado criterios de seguridad en el acceso de los usuarios.
\item Se han utilizado las aplicaciones de forma cooperativa.
\item Se ha elaborado documentación relativa al uso y gestión de las aplicaciones.
\end{enumerate}
\subsection{5. Genera documentos Web utilizando lenguajes de guiones de servidor.}
\label{sec:org0000027}
Criterios de evaluación:
\begin{enumerate}
\item Se han identificado los lenguajes de guiones de servidor más relevantes.
\item Se ha reconocido la relación entre los lenguajes de guiones de servidor y los lenguajes de marcas utilizados en los clientes.
\item Se ha reconocido la sintaxis básica de un lenguaje de guiones concreto.
\item Se han utilizado estructuras de control del lenguaje.
\item Se han definido y utilizado funciones.
\item Se han utilizado formularios para introducir información.
\item Se han establecido y utilizado mecanismos para asegurar la persistencia de la información entre distintos documentos Web relacionados.
\item Se ha identificado y asegurado a los usuarios que acceden al documento Web.
\item Se ha verificado el aislamiento del entorno específico de cada usuario.
\end{enumerate}
\subsection{6. Genera documentos Web con acceso a bases de datos utilizando lenguajes de guiones de servidor.}
\label{sec:org000002a}
Criterios de evaluación:
\begin{enumerate}
\item Se han identificado los sistemas gestores de bases de datos más utilizados en entornos Web.
\item Se ha verificado la integración de los sistemas gestores de bases de datos con el lenguaje de guiones de servidor.
\item Se ha configurado en el lenguaje de guiones la conexión para el acceso al sistema gestor de base de datos.
\item Se han creado bases de datos y tablas en el gestor utilizando el lenguaje de guiones.
\item Se ha obtenido y actualizado la información almacenada en bases de datos.
\item Se han aplicado criterios de seguridad en el acceso de los usuarios.
\item Se ha verificado el funcionamiento y el rendimiento del sistema.
\end{enumerate}
\subsection{7. Realiza modificaciones en gestores de contenidos adaptando su apariencia y funcionalidades.}
\label{sec:org000002d}
Criterios de evaluación:
\begin{enumerate}
\item Se ha identificado la estructura de directorios del gestor de contenidos.
\item Se ha reconocido la funcionalidad de los ficheros que utiliza y su naturaleza (código, imágenes, configuración, entre otros).
\item Se han seleccionado las funcionalidades que hay que adaptar e incorporar.
\item Se han identificado los recursos afectados por las modificaciones.
\item Se ha modificado el código de la aplicación para incorporar nuevas funcionalidades y adaptar otras existentes.
\item Se ha verificado el correcto funcionamiento de los cambios realizados.
\item Se han documentado los cambios realizados.
\end{enumerate}


\subsection{Distribución de RA en unidades de trabajo}
\label{sec:org0000030}

\begin{center}
\begin{tabular}{llllllll}
Peso en la calificación final de cada RA & 5.00\% & 10.00\% & 5.00\% & 5.00\% & 30.00\% & 40.00\% & 5.00\%\\[0pt]
\hline
 & RA1 & RA2 & RA3 & RA4 & RA5 & RA6 & RA7\\[0pt]
\hline
UT1 Servidores web y servidores de aplicaciones & 100.00\% &  &  &  &  &  & \\[0pt]
UT2 Instalación de gestores de contenido &  & 100.00\% &  &  &  &  & \\[0pt]
UT3 Administración de gestores de contenido &  &  & 100.00\% &  &  &  & \\[0pt]
UT4 Generación de páginas dinámicas &  &  &  &  & 45.00\% &  & \\[0pt]
UT5 Acceso a base de datos &  &  &  &  & 45.00\% & 90.00\% & \\[0pt]
UT6 Ofimática Web: owncloud &  &  &  & 100.00\% &  &  & \\[0pt]
UT7 Plugins para gestores de contenido &  &  &  &  & 10.00\% & 10.00\% & 100.00\%\\[0pt]
\end{tabular}
\end{center}


\section{Referencias}
\label{sec:org0000036}
\begin{itemize}
\item Formatos:
\begin{itemize}
\item \href{./iaw-0-contenidos-criterios-evaluacion.reveal.html}{Transparencias}
\item \href{./iaw-0-contenidos-criterios-evaluacion.pdf}{PDF}
\item \href{./iaw-0-contenidos-criterios-evaluacion.wp.html}{Página web}
\item \href{./iaw-0-contenidos-criterios-evaluacion.epub}{EPUB}
\end{itemize}
\item Creado con:
\begin{itemize}
\item \href{https://www.gnu.org/s/emacs/}{Emacs}
\item \href{https://gitlab.com/oer/org-re-reveal}{org-re-reveal}
\item \href{https://www.latex-project.org/}{Latex}
\end{itemize}
\item Alojado en \href{https://alvarogonzalezsotillo.github.io/apuntes-clase}{Github}
\end{itemize}
\end{document}