% Created 2024-09-26 jue 19:15
% Intended LaTeX compiler: pdflatex
\documentclass[a4paper]{article}
\usepackage[utf8]{inputenc}
\usepackage[T1]{fontenc}
\usepackage{graphicx}
\usepackage{longtable}
\usepackage{wrapfig}
\usepackage{rotating}
\usepackage[normalem]{ulem}
\usepackage{amsmath}
\usepackage{amssymb}
\usepackage{capt-of}
\usepackage{hyperref}
\usepackage[spanish]{babel}
\usepackage[usenames,dvipsnames]{color} % Required for custom colors
\renewcommand{\ttdefault}{pcr} % MONOESPACIO CON NEGRIT
\usepackage{lastpage}
\usepackage{listings}
\usepackage{listingsutf8}
\renewcommand{\lstlistingname}{Listado}
\lstset{frame=single,inputencoding=utf8,basicstyle=\scriptsize\ttfamily,showstringspaces=false,numbers=none}
\definecolor{MyDarkGreen}{rgb}{0.0,0.4,0.0} % This is the color used for comments
\lstset{ breaklines=true, postbreak=\mbox{\textcolor{red}{$\hookrightarrow$}\space}, keywordstyle=\bfseries, keywordstyle=[1]\color{Blue}\bfseries,  keywordstyle=[2]\color{Purple}\bfseries,  keywordstyle=[3]\color{Blue}\underbar,   identifierstyle=,   commentstyle=\usefont{T1}{pcr}{m}{sl}\color{MyDarkGreen}\small,   stringstyle=\color{Purple},   showstringspaces=false,   tabsize=2,   morecomment=[l][\color{Blue}]{...} }
\lstset{literate=  {á}{{\'a}}1 {é}{{\'e}}1 {í}{{\'i}}1 {ó}{{\'o}}1 {ú}{{\'u}}1   {Á}{{\'A}}1 {É}{{\'E}}1 {Í}{{\'I}}1 {Ó}{{\'O}}1 {Ú}{{\'U}}1   {à}{{\`a}}1 {è}{{\`e}}1 {ì}{{\`i}}1 {ò}{{\`o}}1 {ù}{{\`u}}1   {À}{{\`A}}1 {È}{{\'E}}1 {Ì}{{\`I}}1 {Ò}{{\`O}}1 {Ù}{{\`U}}1   {ä}{{\"a}}1 {ë}{{\"e}}1 {ï}{{\"i}}1 {ö}{{\"o}}1 {ü}{{\"u}}1   {Ä}{{\"A}}1 {Ë}{{\"E}}1 {Ï}{{\"I}}1 {Ö}{{\"O}}1 {Ü}{{\"U}}1   {â}{{\^a}}1 {ê}{{\^e}}1 {î}{{\^i}}1 {ô}{{\^o}}1 {û}{{\^u}}1   {Â}{{\^A}}1 {Ê}{{\^E}}1 {Î}{{\^I}}1 {Ô}{{\^O}}1 {Û}{{\^U}}1   {œ}{{\oe}}1 {Œ}{{\OE}}1 {æ}{{\ae}}1 {Æ}{{\AE}}1 {ß}{{\ss}}1   {ű}{{\H{u}}}1 {Ű}{{\H{U}}}1 {ő}{{\H{o}}}1 {Ő}{{\H{O}}}1   {ç}{{\c c}}1 {Ç}{{\c C}}1 {ø}{{\o}}1 {å}{{\r a}}1 {Å}{{\r A}}1   {€}{{\euro}}1 {£}{{\pounds}}1 {«}{{\guillemotleft}}1   {»}{{\guillemotright}}1 {ñ}{{\~n}}1 {Ñ}{{\~N}}1 {¿}{{?`}}1 }
\usepackage{caption}
\usepackage{attachfile2}
\usepackage[margin=1.5cm,includeheadfoot,includehead,includefoot]{geometry}
\hypersetup{colorlinks,linkcolor=black}
\usepackage{fancyhdr}
\pagestyle{fancyplain}
\chead{}
\lhead{}
\rhead{}
\cfoot{}
\lfoot{\begin{footnotesize}alvaro.gonzalezsotillo@educa.madrid.org\end{footnotesize}}
\rfoot{\begin{footnotesize}\thepage / \pageref{LastPage}\end{footnotesize}}
\usepackage[skins]{tcolorbox}
\usepackage{multicol}
\usepackage{changepage} %ajdustwidth
\usepackage{fancybox}
\usepackage{attachfile2}
\lhead{Extraordinaria 2021 (es el \\lhead)}
\rhead{Administración y Gestión de Bases de Datos (es el \\rhead)}
\lhead{IAW}
\rhead{Instancia de Wordpress}
\usepackage{svg}
\usepackage{letltxmacro}
\LetLtxMacro{\originalincludegraphics}{\includegraphics}
\renewcommand{\includegraphics}[2][]{\IfFileExists{#2.pdf}{\originalincludegraphics[#1]{#2.pdf}}{\originalincludegraphics[#1]{#2}}}
\LetLtxMacro{\originalincludesvg}{\includesvg}
\renewcommand{\includesvg}[2][]{\IfFileExists{#2.pdf}{\originalincludegraphics[#1]{#2.pdf}}{\originalincludegraphics[#1]{#2.svg.pdf}}}
\usepackage{comment}
\excludecomment{NOTES}
\date{}
\title{Instalación de Wordpress}
\hypersetup{
 pdfauthor={Álvaro González Sotillo},
 pdftitle={Instalación de Wordpress},
 pdfkeywords={},
 pdfsubject={},
 pdfcreator={Emacs 29.4 (Org mode 9.6.15)}, 
 pdflang={Spanish}}
\begin{document}

\maketitle
\setcounter{tocdepth}{1}
\tableofcontents

\captionsetup{font=scriptsize}

\setlength{\parindent}{0em}
\setlength{\parskip}{1em}

\newtcolorbox{Aviso}[1][Aviso]{
  enhanced,
  colback=gray!5!white,
  colframe=gray!75!black,fonttitle=\bfseries,
  colbacktitle=gray!85!black,
  attach boxed title to top left={yshift=-2mm,xshift=2mm},
  title=#1
}

\newtcolorbox{cuadrito}[1][Ignorado]{
  %drop shadow southeast,
  enhanced jigsaw,
  colback=white,
}


\newcommand{\StudentData}{
  \begin{cuadrito}[1\textwidth]
    \vspace{0.3cm}
    \large{
      \textbf{Apellidos:} \hrulefill \\
      \textbf{Nombre:} \hrulefill \\
      \textbf{Fecha:} \hrulefill \hspace{1cm} \textbf{Usuario:} \hrulefill \\
    }
    \vspace{-0.2cm}
  \end{cuadrito}
}

\section{Objetivos de la práctica:}
\label{sec:org0000000}
En esta práctica se espera que el alumno:
\begin{itemize}
\item Utilice un proveedor externo para su pila LAMP
\item Instale Wordpress de forma manual
\item Conozca los niveles de acceso de Wordpress
\end{itemize}

\section{Enunciado}
\label{sec:org0000003}
Una empresa de servicios informáticos desea crear una página web para la autopromoción y  comunicación con sus clientes.

Los contenidos de la página web serán los siguientes:
\begin{itemize}
\item Una presentación de la empresa: qué servicios ofrece, tipos de proyectos (instalación de redes, mantenimiento de ordenadores, instalación de servidores, creación de páginas web)\ldots{}
\item Formas de contacto: perfiles en redes sociales, correo electrónico, teléfono, localización geográfica,\ldots{}
\item Noticias de interés sobre la empresa, para que sean leídas por clientes futuros y actuales
\end{itemize}

En dicha empresa hay un departamento de informática (usuario \texttt{Administrador}), que será el encargado de manejar dicha página web. El jefe de ventas (usuario \texttt{Enrique}) será el responsable de los contenidos de las páginas, pero no los escribirá él directamente, sino que ese trabajo lo llevarán a cabo dos de sus empleados: \texttt{Alicia} y \texttt{Roberto}. Las noticias las escribirán también otros empleados (\texttt{Juan} y \texttt{Susana}), pero tendrán que ser revisadas por el jefe de ventas antes de ser publicadas.

\section{\emph{Hosting}}
\label{sec:org0000006}
Se debe elegir un \emph{hosting} externo para el desarrollo de la práctica. Se sugiere:
\begin{itemize}
\item \url{https://www.awardspace.com/free-web-hosting-registration/}   (no funciona con un email de educamadrid)
\item \url{https://profreehost.com/} (bastante lento)
\item Es válido cualquier otro \emph{hosting} de terceros con
\begin{itemize}
\item Acceso a ficheros FTP, SSH (o interfaz web)
\item Servidor MySQL
\item Ejecución de PHP
\end{itemize}
\end{itemize}

\section{Requisitos}
\label{sec:org0000009}
Se pide que configures un servidor web con Wordpress capaz de servir este sitio, con los requisitos anteriores.
\begin{itemize}
\item Cada tarea deberá realizarla el usuario adecuado para ello (el de menor nivel de acceso)
\item Configurar el servidor, los usuarios, la apariencia general, el logotipo de la empresa,\ldots{}
\item Será necesario también cambiar el tema de wordpress, instalando uno nuevo (por ejemplo, desde \url{https://wordpress.org/themes/browse/community/})
\end{itemize}

El sitio tendrá:
\begin{itemize}
\item Una página de inicio con imágenes y un menú al resto de páginas
\item Una página de \emph{Quiénes somos}
\item Una página con las últimas noticias
\item Habrá al menos tres noticias. Todas tendrán un fichero PDF que podrá descargarse.
\end{itemize}

\section{Normas de entrega}
\label{sec:org000000c}
Se pide la URL de un sitio web con Wordpress instalado, cumpliendo las normas anteriores. Se incluirá también un usuario y contraseña con nivel \emph{Administrador}.

La entrega se realizará en la tarea correspondiente del \href{https://aulavirtual3.educa.madrid.org/ies.alonsodeavellan.alcala/}{aula virtual}.

La autoría del trabajo será individual.
\end{document}
