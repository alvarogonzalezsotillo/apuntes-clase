% Created 2023-11-14 mar 14:00
% Intended LaTeX compiler: pdflatex
\documentclass[a4paper]{article}
\usepackage[utf8]{inputenc}
\usepackage[T1]{fontenc}
\usepackage{graphicx}
\usepackage{longtable}
\usepackage{wrapfig}
\usepackage{rotating}
\usepackage[normalem]{ulem}
\usepackage{amsmath}
\usepackage{amssymb}
\usepackage{capt-of}
\usepackage{hyperref}
\usepackage[spanish]{babel}
\usepackage[usenames,dvipsnames]{color} % Required for custom colors
\renewcommand{\ttdefault}{pcr} % MONOESPACIO CON NEGRIT
\usepackage{lastpage}
\usepackage{listings}
\usepackage{listingsutf8}
\renewcommand{\lstlistingname}{Listado}
\lstset{frame=single,inputencoding=utf8,basicstyle=\scriptsize\ttfamily,showstringspaces=false,numbers=none}
\definecolor{MyDarkGreen}{rgb}{0.0,0.4,0.0} % This is the color used for comments
\lstset{ breaklines=true, postbreak=\mbox{\textcolor{red}{$\hookrightarrow$}\space}, keywordstyle=\bfseries, keywordstyle=[1]\color{Blue}\bfseries,  keywordstyle=[2]\color{Purple}\bfseries,  keywordstyle=[3]\color{Blue}\underbar,   identifierstyle=,   commentstyle=\usefont{T1}{pcr}{m}{sl}\color{MyDarkGreen}\small,   stringstyle=\color{Purple},   showstringspaces=false,   tabsize=2,   morecomment=[l][\color{Blue}]{...} }
\lstset{literate=  {á}{{\'a}}1 {é}{{\'e}}1 {í}{{\'i}}1 {ó}{{\'o}}1 {ú}{{\'u}}1   {Á}{{\'A}}1 {É}{{\'E}}1 {Í}{{\'I}}1 {Ó}{{\'O}}1 {Ú}{{\'U}}1   {à}{{\`a}}1 {è}{{\`e}}1 {ì}{{\`i}}1 {ò}{{\`o}}1 {ù}{{\`u}}1   {À}{{\`A}}1 {È}{{\'E}}1 {Ì}{{\`I}}1 {Ò}{{\`O}}1 {Ù}{{\`U}}1   {ä}{{\"a}}1 {ë}{{\"e}}1 {ï}{{\"i}}1 {ö}{{\"o}}1 {ü}{{\"u}}1   {Ä}{{\"A}}1 {Ë}{{\"E}}1 {Ï}{{\"I}}1 {Ö}{{\"O}}1 {Ü}{{\"U}}1   {â}{{\^a}}1 {ê}{{\^e}}1 {î}{{\^i}}1 {ô}{{\^o}}1 {û}{{\^u}}1   {Â}{{\^A}}1 {Ê}{{\^E}}1 {Î}{{\^I}}1 {Ô}{{\^O}}1 {Û}{{\^U}}1   {œ}{{\oe}}1 {Œ}{{\OE}}1 {æ}{{\ae}}1 {Æ}{{\AE}}1 {ß}{{\ss}}1   {ű}{{\H{u}}}1 {Ű}{{\H{U}}}1 {ő}{{\H{o}}}1 {Ő}{{\H{O}}}1   {ç}{{\c c}}1 {Ç}{{\c C}}1 {ø}{{\o}}1 {å}{{\r a}}1 {Å}{{\r A}}1   {€}{{\euro}}1 {£}{{\pounds}}1 {«}{{\guillemotleft}}1   {»}{{\guillemotright}}1 {ñ}{{\~n}}1 {Ñ}{{\~N}}1 {¿}{{?`}}1 }
\usepackage{caption}
\usepackage{attachfile}
\usepackage[margin=1.5cm,includeheadfoot,includehead,includefoot]{geometry}
\hypersetup{colorlinks,linkcolor=black}
\usepackage{fancyhdr}
\pagestyle{fancyplain}
\chead{}
\lhead{}
\rhead{}
\cfoot{}
\lfoot{\begin{footnotesize}alvaro.gonzalezsotillo@educa.madrid.org\end{footnotesize}}
\rfoot{\begin{footnotesize}\thepage / \pageref{LastPage}\end{footnotesize}}
\usepackage[skins]{tcolorbox}
\usepackage{multicol}
\usepackage{changepage} %ajdustwidth
\usepackage{fancybox}
\usepackage{attachfile2}
\lhead{Extraordinaria 2021 (es el \\lhead)}
\rhead{Administración y Gestión de Bases de Datos (es el \\rhead)}
\lhead{ASGBD}
\rhead{Importar y exportar}
\usepackage{svg}
\usepackage{letltxmacro}
\LetLtxMacro{\originalincludegraphics}{\includegraphics}
\renewcommand{\includegraphics}[2][]{\IfFileExists{#2.pdf}{\originalincludegraphics[#1]{#2.pdf}}{\originalincludegraphics[#1]{#2}}}
\LetLtxMacro{\originalincludesvg}{\includesvg}
\renewcommand{\includesvg}[2][]{\IfFileExists{#2.pdf}{\originalincludegraphics[#1]{#2.pdf}}{\originalincludegraphics[#1]{#2.svg.pdf}}}
\usepackage{comment}
\excludecomment{NOTES}
\date{}
\title{Práctica de backup de una base de datos}
\hypersetup{
 pdfauthor={Álvaro González Sotillo},
 pdftitle={Práctica de backup de una base de datos},
 pdfkeywords={},
 pdfsubject={},
 pdfcreator={Emacs 28.0.50 (Org mode 9.6.9)}, 
 pdflang={Spanish}}
\begin{document}

\maketitle
\captionsetup{font=scriptsize}

\setlength{\parindent}{0em}
\setlength{\parskip}{1em}

\newtcolorbox{Aviso}[1][Aviso]{
  enhanced,
  colback=gray!5!white,
  colframe=gray!75!black,fonttitle=\bfseries,
  colbacktitle=gray!85!black,
  attach boxed title to top left={yshift=-2mm,xshift=2mm},
  title=#1
}

\newtcolorbox{cuadrito}[1][Ignorado]{
  %drop shadow southeast,
  enhanced jigsaw,
  colback=white,
}


\newcommand{\StudentData}{
  \begin{cuadrito}[1\textwidth]
    \vspace{0.3cm}
    \large{
      \textbf{Apellidos:} \hrulefill \\
      \textbf{Nombre:} \hrulefill \\
      \textbf{Fecha:} \hrulefill \hspace{1cm} \textbf{Usuario:} \hrulefill \\
    }
    \vspace{-0.2cm}
  \end{cuadrito}
}
\textattachfile[mimetype=text/plain,author={nba_oracle.sql},description={nba_oracle.sql},subject={nba_oracle.sql}]{../../../gestion-bd-asir1/ejemplosSQL/nba_oracle.sql}{} \vspace{-4em}
\textattachfile[mimetype=text/plain,author={multas-practica.dump},description={multas-practica.dump},subject={multas-practica.dump}]{./multas-practica.dump}{} \vspace{-4em}
\section*{Objetivo de la práctica}
\label{sec:org0000000}
En esta práctica se espera que el alumno:
\begin{itemize}
\item Se familiarice con las utilidades \emph{datapump} de Oracle
\item Realice \emph{backups} y restauraciones de una base de datos
\end{itemize}

Se puede descargar la última versión de este fichero en \href{https://alvarogonzalezsotillo.github.io/apuntes-clase/sistemas-gestores-bbdd-asir2/apuntes/2/asgbd-02-practica-importar-exportar.pdf}{este enlace}.
\section*{(0 puntos) Preparación de la práctica}
\label{sec:org0000003}
\begin{itemize}
\item Crea un usuario con un nombre basado en tus apellidos (por ejemplo, \texttt{GARCIAPEREZ}).
\item Importa en ese usuario el fichero descargable \href{https://alvarogonzalezsotillo.github.io/apuntes-clase/gestion-bd-asir1/ejemplosSQL/nba\_oracle.sql}{\texttt{nba\_oracle.sql}} (también adjunto a este PDF)
\item Registra un \texttt{DIRECTORY} en \texttt{/datos/backups}
\item Crea el usuario \texttt{OPERARIOBACKUP}.
\end{itemize}

Incluye en la memoria todos los comandos utilizados, tanto de Oracle como de Linux.

\section*{(2 puntos) \emph{Backup} del usuario del alumno}
\label{sec:org0000006}
Realiza un \emph{backup} de todas las tablas del usuario utilizando el usuario \texttt{OPERARIOBACKUP}. En la memoria, incluye todos los permisos necesarios y la línea de comandos utilizada.

\section*{(2 puntos) Recupera el \emph{backup}}
\label{sec:org0000009}
\begin{itemize}
\item Borra las filas de la tabla \texttt{PARTIDOS} en las que participe el equipo \texttt{Raptors}.
\item Borra también las filas de la tabla \texttt{PARTIDOS} en las que participe el equipo \texttt{Bulls}
\item Después, recupera solo las 1044 filas de \texttt{Raptors} desde el \emph{backup}, usando el usuario \texttt{OPERARIOBACKUP}. En la memoria, incluye los comandos para borrar las filas y para recuperar el backup.
\item En la base de datos final, debe haber partidos de \texttt{Raptors}, no debe haber partidos de \texttt{Bulls} (excepto contra los \texttt{Raptors}), y los demás partidos no deben estar duplicados.
\end{itemize}

\section*{(3 puntos) Incorpora otro \emph{backup}}
\label{sec:org000000c}
\begin{itemize}
\item Usando el usuario \texttt{OPERARIOBACKUP}, importa el fichero \href{https://alvarogonzalezsotillo.github.io/apuntes-clase/sistemas-gestores-bbdd-asir2/apuntes/2/multas-practica.dump}{\texttt{multas-practica.dump}} en el tu usuario.
\item Incluye en la memoria los comandos exactos utilizados.
\end{itemize}

\section*{(3 puntos) \emph{Backup} de algunas tablas}
\label{sec:org000000f}
Realiza un \emph{backup} de las tablas \texttt{JUGADORES}, \texttt{MATRICULAS} y \texttt{EQUIPOS}. Incluye en la memoria los comandos utilizados y el fichero de backup generado.  

\section*{Qué se valorará}
\label{sec:org0000012}
\begin{itemize}
\item La corrección técnica de los comandos utilizados
\item La claridad en la explicación de dichos comandos
\item Que no se concedan más permisos/privilegios de los necesarios a los usuarios / ficheros /directorios creados
\end{itemize}

\section*{Normas de entrega}
\label{sec:org0000015}
Se entregarán en un fichero ZIP:
\begin{itemize}
\item Los comandos utilizados para realizar la práctica. No es necesario un documento al uso, es suficiente un fichero de texto, siempre que esté claro y bien formateado.
\item El archivo de \emph{backup} de las tablas \texttt{JUGADORES}, \texttt{MATRICULAS} y \texttt{EQUIPOS}, y los archivos de \emph{log} generado en el prodceso
\end{itemize}

La entrega se realizará en la tarea correspondiente del \href{https://aulavirtual3.educa.madrid.org/ies.alonsodeavellan.alcala}{aula virtual}.

La autoría del trabajo será individual.
\end{document}