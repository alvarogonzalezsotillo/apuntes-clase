% Created 2023-10-17 mar 13:37
% Intended LaTeX compiler: pdflatex
\documentclass[a4paper]{article}
\usepackage[utf8]{inputenc}
\usepackage[T1]{fontenc}
\usepackage{graphicx}
\usepackage{longtable}
\usepackage{wrapfig}
\usepackage{rotating}
\usepackage[normalem]{ulem}
\usepackage{amsmath}
\usepackage{amssymb}
\usepackage{capt-of}
\usepackage{hyperref}
\usepackage[spanish]{babel}
\usepackage[usenames,dvipsnames]{color} % Required for custom colors
\renewcommand{\ttdefault}{pcr} % MONOESPACIO CON NEGRIT
\usepackage{lastpage}
\usepackage{listings}
\usepackage{listingsutf8}
\renewcommand{\lstlistingname}{Listado}
\lstset{frame=single,inputencoding=utf8,basicstyle=\scriptsize\ttfamily,showstringspaces=false,numbers=none}
\definecolor{MyDarkGreen}{rgb}{0.0,0.4,0.0} % This is the color used for comments
\lstset{ breaklines=true, postbreak=\mbox{\textcolor{red}{$\hookrightarrow$}\space}, keywordstyle=\bfseries, keywordstyle=[1]\color{Blue}\bfseries,  keywordstyle=[2]\color{Purple}\bfseries,  keywordstyle=[3]\color{Blue}\underbar,   identifierstyle=,   commentstyle=\usefont{T1}{pcr}{m}{sl}\color{MyDarkGreen}\small,   stringstyle=\color{Purple},   showstringspaces=false,   tabsize=2,   morecomment=[l][\color{Blue}]{...} }
\lstset{literate=  {á}{{\'a}}1 {é}{{\'e}}1 {í}{{\'i}}1 {ó}{{\'o}}1 {ú}{{\'u}}1   {Á}{{\'A}}1 {É}{{\'E}}1 {Í}{{\'I}}1 {Ó}{{\'O}}1 {Ú}{{\'U}}1   {à}{{\`a}}1 {è}{{\`e}}1 {ì}{{\`i}}1 {ò}{{\`o}}1 {ù}{{\`u}}1   {À}{{\`A}}1 {È}{{\'E}}1 {Ì}{{\`I}}1 {Ò}{{\`O}}1 {Ù}{{\`U}}1   {ä}{{\"a}}1 {ë}{{\"e}}1 {ï}{{\"i}}1 {ö}{{\"o}}1 {ü}{{\"u}}1   {Ä}{{\"A}}1 {Ë}{{\"E}}1 {Ï}{{\"I}}1 {Ö}{{\"O}}1 {Ü}{{\"U}}1   {â}{{\^a}}1 {ê}{{\^e}}1 {î}{{\^i}}1 {ô}{{\^o}}1 {û}{{\^u}}1   {Â}{{\^A}}1 {Ê}{{\^E}}1 {Î}{{\^I}}1 {Ô}{{\^O}}1 {Û}{{\^U}}1   {œ}{{\oe}}1 {Œ}{{\OE}}1 {æ}{{\ae}}1 {Æ}{{\AE}}1 {ß}{{\ss}}1   {ű}{{\H{u}}}1 {Ű}{{\H{U}}}1 {ő}{{\H{o}}}1 {Ő}{{\H{O}}}1   {ç}{{\c c}}1 {Ç}{{\c C}}1 {ø}{{\o}}1 {å}{{\r a}}1 {Å}{{\r A}}1   {€}{{\euro}}1 {£}{{\pounds}}1 {«}{{\guillemotleft}}1   {»}{{\guillemotright}}1 {ñ}{{\~n}}1 {Ñ}{{\~N}}1 {¿}{{?`}}1 }
\usepackage{caption}
\usepackage{attachfile}
\usepackage[margin=1.5cm,includeheadfoot,includehead,includefoot]{geometry}
\hypersetup{colorlinks,linkcolor=black}
\usepackage{fancyhdr}
\pagestyle{fancyplain}
\chead{}
\lhead{}
\rhead{}
\cfoot{}
\lfoot{\begin{footnotesize}alvaro.gonzalezsotillo@educa.madrid.org\end{footnotesize}}
\rfoot{\begin{footnotesize}\thepage / \pageref{LastPage}\end{footnotesize}}
\usepackage[skins]{tcolorbox}
\usepackage{multicol}
\usepackage{changepage} %ajdustwidth
\usepackage{fancybox}
\usepackage{attachfile2}
\lhead{Extraordinaria 2021 (es el \\lhead)}
\rhead{Administración y Gestión de Bases de Datos (es el \\rhead)}
\lhead{ASGBD}
\rhead{Usuarios y niveles de acceso}
\usepackage{svg}
\usepackage{letltxmacro}
\LetLtxMacro{\originalincludegraphics}{\includegraphics}
\renewcommand{\includegraphics}[2][]{\IfFileExists{#2.pdf}{\originalincludegraphics[#1]{#2.pdf}}{\originalincludegraphics[#1]{#2}}}
\LetLtxMacro{\originalincludesvg}{\includesvg}
\renewcommand{\includesvg}[2][]{\IfFileExists{#2.pdf}{\originalincludegraphics[#1]{#2.pdf}}{\originalincludegraphics[#1]{#2.svg.pdf}}}
\usepackage{comment}
\excludecomment{NOTES}
\date{}
\title{Práctica de usuarios y niveles de acceso}
\hypersetup{
 pdfauthor={Álvaro González Sotillo},
 pdftitle={Práctica de usuarios y niveles de acceso},
 pdfkeywords={},
 pdfsubject={},
 pdfcreator={Emacs 28.0.50 (Org mode 9.6.9)}, 
 pdflang={Spanish}}
\begin{document}

\maketitle
\tableofcontents

\captionsetup{font=scriptsize}

\setlength{\parindent}{0em}
\setlength{\parskip}{1em}

\newtcolorbox{Aviso}[1][Aviso]{
  enhanced,
  colback=gray!5!white,
  colframe=gray!75!black,fonttitle=\bfseries,
  colbacktitle=gray!85!black,
  attach boxed title to top left={yshift=-2mm,xshift=2mm},
  title=#1
}

\newtcolorbox{cuadrito}[1][Ignorado]{
  %drop shadow southeast,
  enhanced jigsaw,
  colback=white,
}


\newcommand{\StudentData}{
  \begin{cuadrito}[1\textwidth]
    \vspace{0.3cm}
    \large{
      \textbf{Apellidos:} \hrulefill \\
      \textbf{Nombre:} \hrulefill \\
      \textbf{Fecha:} \hrulefill \hspace{1cm} \textbf{Usuario:} \hrulefill \\
    }
    \vspace{-0.2cm}
  \end{cuadrito}
}





\section*{Objetivo de la práctica}
\label{sec:org0000000}
En esta práctica utilizaremos a la base de datos \textbf{Oracle} como un verdadero servidor, conectándonos desde otros ordenadores. Para ello:
\begin{itemize}
\item Crearemos tablas en un \emph{tablespace} separado
\item Crearemos un usuario para cada compañero de clase
\item Asignaremos privilegios utilizando un rol
\end{itemize}

La última versión de este documento está accesible en \href{https://alvarogonzalezsotillo.github.io/apuntes-clase/sistemas-gestores-bbdd-asir2/apuntes/1/asgbd-1-practica-usuarios-permisos-ii.pdf}{este enlace}
\section{(1 punto) Creación de un \emph{tablespace}}
\label{sec:org0000003}

Crea un tablespace
\begin{itemize}
\item De nombre \texttt{PARAOTROS}.
\item Con dos ficheros que se guardarán en \texttt{/datos/paraotros}. (50\%)
\item El tamaño de cada fichero será como máximo de 100 MBytes. (50\%)
\end{itemize}

\section{(2 puntos) Crear usuarios para tus compañeros}
\label{sec:org0000006}

Crea un usuario para tí, uno para cada uno de tus compañeros, y uno para el profesor. La contraseña inicial será la misma que el nombre (mayúsculas), excepto en tu usuario que debería ser una contraseña secreta:
\begin{multicols}{3}
\begin{itemize}
\item RUBENAVILA3
\item JORGEBAS
\item DOMINIKCISLO
\item JESUSCUEVAS4
\item DAVIDCZARNECKI
\item EMMANUELFERNAN
\item NOELIAFERNANDE
\item JOSELOPEZ16
\item JESUSMUNOZ7
\item MIGUELMUYO
\item HUGOPABLO
\item CREINA
\item ALEJANDRORIVIL
\item IRUBIOMARTIN
\item ALVAROVICENTE1
\item ALVAROGONZALEZ
\end{itemize}
\end{multicols}
\textbf{Nota}: Son muchos usuarios, así que es aconsejable utilizar un \emph{script} en vez de crearlos manualmente.

Tu propio usuario tendrá como \emph{tablespace} por defecto \texttt{USERS}, y los demás \texttt{PARAOTROS} (33\%)
Los usuarios necesitan poder conectarse a la base de datos y crear tablas en su \emph{tablespace} por defecto (33\%)
Los usuarios de los otros alumnos tendrán una cuota de 10 MByte en \texttt{PARAOTROS}, y no podrán insertar datos en \texttt{USERS} (33\%).

Cuando tengas listos los usuarios:
\begin{itemize}
\item Los datos de conexión a cada servidor \href{https://docs.google.com/spreadsheets/d/17A6TpGOixGVVZr1VlocN6ZhMuKqmQjD3XedyIne4ows/edit\#gid=0}{se indicarán en esta hoja excel}
\item El resto de usuarios podrán conectarse con \texttt{sqlplus USUARIO/USUARIO@HOST:1521/SIDOSERVICIO}
\item Pide a algunos compañeros que comprueben su usuario
\item Pide al profesor que compruebe su usuario
\end{itemize}

\section{(2 puntos) Creación de las tablas}
\label{sec:org0000009}
Cambia la contraseña de tu usuario, si no lo has hecho ya, para que ningún compañero pueda utilizarlo (con \texttt{ALTER USER}).

\begin{itemize}
\item Crea un \emph{tablespace} de nombre \texttt{CARRERAS}, con un \emph{datafile} en el directorio \texttt{/datos/carreras} (20\%)
\item Con tu propio usuario, crea las tablas en ese \emph{tablespace} (80\%)
\begin{itemize}
\item Utiliza el \href{https://raw.githubusercontent.com/alvarogonzalezsotillo/apuntes-clase/gh-pages/sistemas-gestores-bbdd-asir2/apuntes/1/sql/carreras-coches.sql}{\emph{script} \texttt{carreras-coches.sql}} para la creación de las tablas
\item Tendrás que modificar el \emph{script} para que tenga en cuenta el nuevo \emph{tablespace}
\item Tu usuario necesitará cuota en ese \emph{tablespace}
\end{itemize}
\end{itemize}

\section{(3 puntos) Dar acceso a otros usuarios a un campo de tus tablas}
\label{sec:org000000c}
Tras la importación, haz que las tablas puedan ser leídas por el resto de usuarios:
\begin{itemize}
\item Haz que el resto de usuarios pueda realizar \texttt{SELECT} sobre tus tablas. (20\%)
\item Crea sinónimos en todos los usuarios para que puedan acceder a tus tablas sin problemas (20\%)
\begin{itemize}
\item Por ejemplo, el usuario \texttt{profesor} debería poder ejecutar \texttt{SELECT * FROM CIRCUITOS}, puesto que habrás creado un sinónimo del tipo \texttt{CREATE PUBLIC SYNONYM CIRCUITOS FOR MIUSUARIO.CIRCUITOS}.
\end{itemize}
\end{itemize}

\begin{Aviso}
Son muchos usuarios. Agrúpalos de algún modo. (40\%)
\end{Aviso}

Después, haz que puedan escribir en un campo de la tabla \texttt{RESULTADOS}:
\begin{itemize}
\item Crea un nuevo campo en la tabla \texttt{RESULTADOS}: \texttt{VALORACION}, de tipo \texttt{VARCHAR(20)}.
\item Da permisos al resto de usuarios para poder leer y actualizar este campo, con una orden \texttt{GRANT} (20\%)
\item Pide a algún otro compañero que compruebe que funciona
\item Pide al profesor que compruebe que funciona.
\end{itemize}


\section{(0.5 puntos) \texttt{PROFESOR} con permisos especiales}
\label{sec:org000000f}
Haz un usuario \texttt{PROFESOR} con contraseña \texttt{PROFESOR} que pueda iniciar sesión y tenga permisos de lectura en las vistas de sistema
\begin{itemize}
\item \texttt{DBA\_COL\_PRIVS}
\item \texttt{DBA\_DATA\_FILES}
\item \texttt{DBA\_ROLES}
\item \texttt{DBA\_ROLE\_PRIVS}
\item \texttt{DBA\_SYNONYMS}
\item \texttt{DBA\_TAB\_COLS}
\item \texttt{DBA\_TAB\_PRIVS}
\item \texttt{DBA\_TABLES}
\item \texttt{DBA\_TS\_QUOTAS}
\item \texttt{DBA\_USERS}
\end{itemize}


\begin{Aviso}
Sin el usuario \texttt{PROFESOR} y los permisos especificados no se podrá corregir esta práctica
\end{Aviso}
\section*{Instrucciones de entrega}
\label{sec:org0000012}
\begin{itemize}
\item Esta práctica se corrige de forma semiautomática.
\item El profesor se conectará a cada servidor de base de datos a comprobar la configuración en el día indicado.
\end{itemize}
\end{document}